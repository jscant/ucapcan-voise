\hypertarget{group__getVDOp}{}\section{get\+V\+D\+Op}
\label{group__getVDOp}\index{get\+V\+D\+Op@{get\+V\+D\+Op}}


This is a M\+EX function, and as such the inputs and outputs are constricted to the following\+:  


This is a M\+EX function, and as such the inputs and outputs are constricted to the following\+: 


\begin{DoxyParams}{Parameters}
{\em nlhs} & Number of outputs \\
\hline
{\em plhs} & Pointer to outputs \\
\hline
{\em nrhs} & Number of inputs \\
\hline
{\em prhs} & Pointer to inputs\\
\hline
\end{DoxyParams}
In Matlab, this corresponds to the following parameters and outputs\+: 
\begin{DoxyParams}{Parameters}
{\em VD} & Voronoi diagram struct \\
\hline
{\em W} & Matrix of pixel values \\
\hline
{\em metric\+ID} & Key from 1-\/6 indicating which metric to use\+: (1) Median (2) Mean (3) Range (4) Square root of the number of pixels (5) Normalised range (6) Standard deviation \\
\hline
{\em mult} & (For metric\+ID = 5 and 6 only) Multiplier of each pixel. For metric\+ID = 5, coefficient is 1/mult. For metric\+ID = 6, coefficient is equal to mult. \\
\hline
\end{DoxyParams}
\begin{DoxyReturn}{Returns}
Sop Vector of metric value for each Voronoi region 

Wop Matrix of metric values for each pixel. All pixels in same voronoi region have same Wop value. Pixels equidistant from two or more closest seeds (with $ \nu_{ij} $ = 1) have Wop $_{ij} $ = NaN, as per the Matlab implementation. 
\end{DoxyReturn}
